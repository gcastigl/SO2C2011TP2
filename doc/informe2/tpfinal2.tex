% Preview source code

%% LyX 2.0.0 created this file.  For more info, see http://www.lyx.org/.
%% Do not edit unless you really know what you are doing.
\documentclass[english]{article}
\usepackage[T1]{fontenc}
\usepackage[latin9]{inputenc}
\usepackage{babel}
\begin{document}

\section{PIPES}

El principal obstáculo de la implementación fue decidir donde escribir
el mensaje enviado por el escritor a los receptores. Como primera
alternativa se pensó en guardarlo en el mismo fifo, dado que es un
file y puede tener contenido. Si bién esta era una implementación
es sencilla, hacía que la transmisión del mensaje sea extremadamente
lenta (lectura y escritua a disco). Para superar esto se optó por
guardar el mensaje en una zona de memoria especialmente destinada
para esto. La principal limitación es que se pueden abrir sólo un
número fijo de fifos al mismo tiempo. Al estar el mensaje en memoria,
la transmisión es mucho más rápida. Otro problema que traen los fifos
es que deben ser bloqueados tanto para lectura como para escritura.
Se utilizaron los semáforos dados en el ejemplo mtask1 de la cátedra
para esto. Se crearon los comandos mkfifo para crear el fifo y pitest
para realizar pruebas cobre el mismo.

El comando pitest crea el fifo con nombre \textquotedblleft{}test.pipe\textquotedblright{}
si no existe, en caso contrario se conecta al mismo. Y bloquea al
proceso para el modo en que se conectó al pipe. Pitest w: modo escritura.
Pitest r: modo lectura. NOTA: ambos procesos deben estar en el mismo
directorio dado que el fifo se crea en el directorio actual.

CACHE Gran parte de la implementación del cache fue realizada (y detallada)
durante el tp anterior. Sin embargo se mejoraron algunos aspectos
en el algoritmo de reemplazo de sectores (Least Recently Used). Antes
se tenía un contador por cada sector del disco y cada vez que era
accedido ya sea para lectura o escritura, era incrementado. El problema
de esto es que sectores eran ordenados por cantidad de accesos y podían
persistir en cache por mucho tiempo hasta que otro file tuviera mayor
o igual cantidad. Esto traía el problema de que los sectores más accedidos
nunca eran removidos del cache. En la nueva implementación se optó
por agregar un contador de edad en cache para que si un sector cacheado
supera la edad tope, se elimine.


\section{FILE SYSTEM}

Se implementó el manejo de inodos por medio de un bitmap, por sobre
el bitmap de contenidos que ya existía, lo que permitió la implementación
como mv y rm.

Se agregaron los comandos standard de linux que faltaban en el tp
anterior. Mv desde hasta: mueve el archivo desde a hasta. Rm nombre:
elimina el archivo nombre. cp desde hasta: copia el archivo desde
a hasta. Ls: se mejoró la forma de mostrar archivos, agregando colores
y caracteres especiales para diferenciar los tipos (ej: $link@$ )


\section{PROCESOS / SCHEDULING}

Se le agregaron a los procesos una lista de file descriptors para
mantener un registro de los archivos abiertos por el mismo. Cuando
un proceso se bloquea se le setea un flag para saber a causa de qué
se bloqueo. Puede ser bloqueado para input, fifos, wait-child, wait-semaphore.
Gracias a estos cambios se pudo remover el busy-waiting en todos los
procesos, en especial, la shell que permanece bloqueada por input. 


\section{PAGING}


\subsection{Virtual Address}

Se implementaron directorios y tablas de páginas. Se dividió la memoria
de a $4K$ ($0x1000$). Cuando se inicializa paging se crea un heap
de kernel en la dirección $0xC00000000$. En una primera implementación
los procesos creaban sus stacks con malloc, estro traía el problema
de que el stack estaba dentro del heap, lo que limitaba mucho la memoria.
Se optó por mover los stacks de los procesos a una zona de memoria
dedicada para esto, cerca del final del address space. De esta manera
al tener el heap arriba que avanza en forma decreciente y el stack
abajo que avanza en forma creciente, la memoria queda mucho más organizada.
Cada vez que se crea un proceso se alocan dos páginas de memoria.
No obstante adelantando que el proceso seguramente pida más páginas
los nuevos procesos no son creados inmediatamente después sino que
se dejan un número de páginas en blanco entre los dos. 


\subsection{Physical Address}

Cuando hablamos de páginas, estamos hablando de direcciones virtuales
de memoria, una página puede querer ir a la dirección de memoria $0x8000000$
($128MB$) pero para que esto funcione en nuestra máquina virtual
que le dimos $16MB$ de memoria necesita hacer un mapeo de la misma.
Los bloques reales de memoria los llamamos frames y son manejados
mediante un bitmap.


\subsection{Heap}

El algoritmo que maneja el heap utiliza dos conceptos: blocks \& holes.
Llamamos blocks a aquellas zonas de memoria que contienen información
(mallocs no liberados) y holes a las zonas de memoria libre. En un
principio la memoria del heap es un gran hole. Por cada hole existe
un indice en la tabla ordenada de holes. El orden es necesario porque
luego buscaremos los espacios mínimos a ocupar a la hora de hacer
mallocs. Tantos los blocks como los holes tienen un header y un footer
para guardar información de la porción de memoria a la cual pertenecen.
A medida que se hacen mallocs en el sistema, el heap puede ir quedando
chico, para poder detectar esto se agregaron una serie de assertions
para poder expandir las páginas reservadas para el heap. También se
agregó la posibilidad de que se contraiga en caso de que se libere
la misma.


\end{document}

